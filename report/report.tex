\documentclass[a4paper]{article}
\usepackage[utf8x]{inputenc}
\usepackage[T1]{fontenc}
\usepackage[french]{babel}

\author{Gautier \textsc{Colajanni}, Cédric \textsc{Jezequel},\\ Julien \textsc{Marcou}, Pierre \textsc{Poilane}, Paul \textsc{Rivière}, Kevin \textsc{Thek}}

\title{Rapport de projet Acquisition de Connaissances \\ Authorship Attribution}

\begin{document}

\maketitle

\section{Introduction}

\section{Méthode 1 : Term-frequency criterion}
Une première méthode consiste à classifier un document en fonction de la présence de certains mots dans ce texte. Il va s'agir ainsi de construire un classifieur en utilisant un ensemble d'apprentissage, et de trouver la classe qui correspond au mieux à un document faisant partie de l'ensemble de test.

\section{Méthode 2 : ---}

\section{Comparaison de performance}

\end{document}